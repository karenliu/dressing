\section{Introduction}

% \karen{Very high level comments: 1. Most problems mentioned in the
%   second paragraph are based on the assumption that animators use
%   physics simulation to generate cloth motion. We might want to
%   establish this assumption first by describing a general or
%   reasonable approach for creating cloth/human motion. 2. We want to
%   explicitly state the input of our system. 3. Our problem can be
%   viewed as path planning. It's very difficult because the cloth keeps
%   moving and the human body can easily self-collide. Further, we are
%   facing a very unique problem regarding human/cloth collision. Unlike
%   most navigation problems, our goal is not to avoid colliding with
%   cloth. Instead, our goal is to embrase and utilize contacts to
%   change the state of clothes in order to achieve the task. }

This paper describes a system for animating the activity of putting on
clothing.  Dressing is one of the most common activities that each of us
carries out each day.  Scenes of dressing are also common in live-action
movies and television.  Some of these scenes are iconic, such as the
``jacket on, jacket off'' drill in The Karate Kid (2010 version) or
Spiderman pulling his mask over his head for the first time.  Such
dressing scenes are noticeably absent in computer animated films.  Despite
the importance of dressing in our lives and in film, there is as yet no
systematic approach to animating a human that is putting on clothing.

% Mr. Rogers putting on his cardigan sweater in the children’s TV show,
% Mrs.  Robinson slipping on her stockings in The Graduate.
%
% The trailer for The Incredibles shows how far a computer animator will
% go to avoid showing a character pulling on their clothes.  

Our goal is to provide a system that will allow an animator to create
motion for a human character that is dressing.  We want the animator to
have a high degree of control over the look of the final animation.  To
this end, we desire a system that allows the user to describe the dressing scene as a sequence
of high-level actions.  Also, we would like our system to
accept approximated human motion, in either the form of keyframes or
motion capture, as reference for styles or aesthetics.  Thus the input from
the animator for a given dressing scene consists of: a character model,
a garment, a sequence of dressing actions, and reference
motions for the character. In order to create animation that is
physically plausible, we made the choice to use physical simulation of
cloth to guide the garment motions. By using cloth simulation, the
human figure, made of a collection of rigid segments, can interact
with the cloth in a natural manner.
 
% Our system takes these inputs from the animator and creates a physically
% realistic looking animation.  In order to create animation that is
% physically plausible, we made the choice to use physical simulation of
% cloth to guide the garment motions.  By using cloth simulation, the motion
% of the character affects the cloth motion in a natural manner.  We
% represent our human figures as a collection of rigid segments that are
% connected by joints.

% People use a wide variety
% of clothing types, including: shirts, pants, underwear, dresses, skirts,
% socks, shoes, hats, gloves, vests, jackets, and sweaters.

The essence of animating the act of dressing is modeling the
interaction between the human character and the cloth.  The human's motion
must adapt to the motion of the cloth, otherwise problems occur such as the
clothing slipping off or a hand getting stuck in a fold.  We often take
for granted the complex set of motions that are needed to put on our
clothes.  The seemingly simple act of putting on a jacket requires a
careful coordination between the person and the jacket.  Unconsciously we
make constant adjustments to our hand’s position when inserting it into
the jacket’s sleeve.  We hold our body at an angle to keep a sleeve from
sliding off our shoulder.  After putting on the first sleeve, we may use
any of several strategies to get our hand behind our back and within reach
of the second sleeve.  A system for animation of dressing must address
these kinds of complexities.

We have found that a small set of \emph{primitive actions} account for the vast
majority of the motions that a person goes through to put on an article of
clothing.  The approach that we take to dressing is to first have the
animator divide a desired dressing motion into a small number of such
actions.  These actions include placing a hand or foot through an opening,
pulling the garment onto a limb, and stretching out a limb after it has
been positioned in the clothing.  Once this sequence of actions has been
assembled, producing the dressing animation can proceed.  The system steps
forward in time, updating the cloth's position through simulation.  The
character's motion during each of the actions are guided by optimization
and planning in order to satisfy the requirements of a given
action. The system adjusts the character's pose to match the end of one
action to the start of the next. Some portions of a dressing sequence do
not require the character to react to the cloth, and such segments can follow
the provided keyframe or motion capture data.

To a large degree, the problem that a dressing animation system must solve
is a form of \emph{path planning}.  However, the dressing problem has a
few unique challenges which are not addressed by standard path planning
algorithms. First, the character's body parts must move in coordination to
complete the task while preventing self-intersection. Moreover, the
character must move in and around the cloth in such a way that the garment
ends up properly on the person. Unlike typical path planning that avoids
collisions, contact between the body parts and the cloth is to be
expected. In fact, \emph{utilizing} contact to expand the opening of a folded
sleeve or fixate a part of cloth on the body is crucial for
successful dressing.

% To a large degree, the problem that a dressing animation system must solve
% is a form of \emph{path planning}.  First, the character's motion must not
% cause self-intersection between body parts.  We use several techniques
% for avoiding such collisions between body parts.  Moreover, the character
% must move in and around the cloth in such a way that the garment ends up
% properly on the person.  Some actions specifically require the character
% to react to the motion of the cloth.  Consider the case of trying to put a
% foot into the opening for one leg in a pair of pants.  Our system
% constantly adjusts the motion of the character's foot so that it draws
% closer to the opening without getting tangled in the cloth.  We accomplish
% this by using visibility information about the cloth to provide feedback
% that controls the limb's motion.  Note that this is not the typical form
% of path planning that would avoid colliding with the cloth.  Contact
% between the person and the cloth is to be expected.

Using our action-based dressing system, we have produced a variety of
animations of a character that is putting on various types of clothes.
This includes putting on a jacket, pulling on pants while sitting, putting
on pants while standing, dynamically swinging on a vest, and having one character assist another in putting on a robe.


