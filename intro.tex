\section{Introduction}

This paper describes a system for animating the activity of putting on
clothing.  Dressing is one of the most common activities that each of us
carries out each day.  Scenes of dressing are also common in live-action
movies and television.  Some of these scenes are iconic, such as Mr.
Rogers putting on his cardigan sweater in the children’s TV show,
Spiderman pulling his mask over his head for the first time, or Mrs.
Robinson slipping on her stockings in The Graduate.  Such dressing scenes
are noticeably absent in computer animated films.  Despite the importance
of dressing in our lives and in film, there is as yet no systematic
approach to animating a human that is putting on clothing.

% The trailer for The Incredibles shows how far a computer animator will
% go to avoid showing a character pulling on their clothes.  

The essence of animating the action of dressing is modeling the
interaction between the human character and the cloth.  The human's motion
must adapt to the cloth motion, otherwise problems occur such as the
clothing slipping off or a hand getting stuck in a fold.  We often take
for granted the complex set of motions that are needed to put on our
clothes.  The seemingly simple act of putting on a jacket requires a
careful coordination between the person and the jacket.  Unconsciously we
make constant adjustments to our hand’s position when inserting it into
the jacket’s sleeve.  We hold our body at an angle to keep a sleeve from
sliding off our shoulder.  After putting on the first sleeve, we may use
any of several strategies to get our hand behind our back and within reach
of the second sleeve.  A system for animation of dressing must address
these kinds of complexities.

Our goal is to provide a system that will allow an animator to create
motion for a human character that is dressing.  People use a wide variety
of clothing types, including: shirts, pants, underwear, dresses, skirts,
socks, shoes, hats, gloves, vests, jackets, and sweaters.  Most of these
articles are made of cloth, so we use a cloth simulator to calculate the
motion of the clothes.  We use the typical representation of a human
character as a collection of rigid segments that are connected by joints.
To guide the human figure's motion, we use optimization and path planning.

We have found that a small set of \emph{typical actions} account for the vast
majority of the motions that a person goes through to put on an article of
clothing.  The approach that we take to dressing is to first have the
animator divide a given dressing motion into a small number of such
actions.  These actions include placing a hand or foot through an opening,
pulling the clothing onto a limb, and re-orienting a limb after it has
been positioned in the clothing.  Once this sequence of actions has been
assembled, producing the dressing animation can proceed.  The system steps
forward in time, updating the cloth's position through simulation.  The
character's motion during each of the actions are guided by optimization
and path planning in order to satisfy the requirements of a given
action.  The system adjusts the character's pose to match the end of one
action to the start of the next.  Some portions of a dressing sequence do
not require the character to react to the cloth, and such parts can follow
motion capture data or keyframes.

During a typical action, the system may be required to find a motion that
avoids self-collisions between body parts.  This requires path planning,
and we use the RRT algorithm to plan such motions.  In addition, in most
actions the character's body must respect several constraints, including
hard constraints such as joint angle limits and soft constraints such as
following a planned path for a hand or a foot.  We make use of
optimization to adhere to such constraints.  Some actions specifically
require the character to react to the motion of the cloth.  Consider the
case of trying to put a foot into the opening for one leg in a pair of
pants.  Our system constantly adjusts the motion of the character's foot
so that it draws closer to the opening without getting tangled in the
cloth.  This is accomplished by following a vector field across the
cloth's surface that acts as a guide for the foot.

Using this action-based system, we have produced a variety of animations
of a character that is putting on various types of clothes.  This includes
putting on a jacket, pulling on pants while sitting, putting on pants
while standing, and having one character assist another in putting on a
robe.


