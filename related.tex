\section{Related Work}

Close range interacting with surrounding objects or humans is an important research problem in character animation. The problem is challenging because it often involves potentially conflicting goals: maintaining intentional spatial constraints while avoiding unintentional contacts. Much research has been focusing on the challenge of handling contact and spatial constraints between body parts or objects \cite{Gleicher:1998:RMN,Liu:2006:CCO,Ho:2009:CMS,Kim:2009:SMM,Ho:2010:SRP}. Ho \etal \shortcite{Ho:2010:SRP} used an ``interaction mesh'' to encode the spatial relationship of interacting body parts. By minimizing the local deformation of the mesh, their method preserved the desired spatial constraints while reducing unintentional contacts or interpenetrations. In additional to contact problem, close range interaction also demands sophisticated path planning. Previous work exploited inverse kinematics and motion planning techniques to generate motion that satisfies desired manipulation tasks in complex or cluttered environments \cite{Kallmann:2003:PCF,Yamane:2004:SAH}. A large body of robotics literature on the topic of motion planning for full-body manipulation is also highly relevant to synthesis of close range interaction \cite{Harada:2003:PMH,Takubo:2005:PAO,Yoshida:2005:HMP,Nishiwaki:2006:MCS}. In this paper, self-dressing is also an example of close range interaction. Unlike most problems studied previously, self-dressing involves interacting with a unique object, cloth, which is highly deformable with frequent self-collisions. 


% Depending on the applications, previous research developed techniques for different subproblems, such as collisions handling \cite{Ye:2012}, spatial constraints solving \cite{Ho:2010:SRP}, and path planning \cite{Kallman:2003,Yamane:2004:SAH,Bai:2012:SCO}. These existing techniques generated interesting animations beyond what simple forward simulation or keyframe interpolation could achieve. However, most methods assumed that the character interacts with humans or objects made of rigid bodies with a few exceptions. Ho and Kumura \shortcite{Ho:2009:CMS} introduced a technique to interact with deformable bodies using topology coordinates, in which the topological relationship of the character's body and the environment can be easily controlled. \shortcite{Jain:2010:SCT} proposed a simple 

% - Spatial Relationship Preserving Character Motion Adaptation
% Path planning
% - Synthesis of Concurrent Object Manipulation Tasks
% - Planning collision-free reaching motions for interactive object manipulation and grasping
% - Synthesizing animations of human manipulation tasks


% Hand manipulation with soft objects
% - Coupling Cloth and Rigid Bodies for Dexterous Manipulation
%  Robotic cloth manipulation

Researchers studying dexterous manipulation have developed control algorithms to handle different types of manipulation, such as grasping \cite{Pollard:2005:PBG,Kry:2006:ICS,Wang:2013:VHM,Zhao:2013:RRP}, finger gaiting \cite{Ye:2012:SDH}, or rolling \cite{Bai:2014:DMU}. These methods can successfully manipulate rigid bodies with various sizes and masses, but it is not clear whether they can be extended to manipulating deformable bodies, which typically have more degrees of freedom than rigid bodies. In contrast to computer graphics, manipulating deformable bodies has been addressed extensively in robotics. Researchers have demonstrate robots manipulating cloth, ropes, cables, foam rubber, or sheet metal \cite{Kosuge:1995:MFO,Wu:1995:AHC,Fahantidis:1997:RHF,Osawa:2007:UML,Cusumano:2011:BCD,Bersch:2011:BRC,Miller:2012:GAR}. Our work is related to manipulation of cloth for folding laundry \cite{Osawa:2007:UML,Cusumano:2011:BCD,Bersch:2011:BRC,Miller:2012:GAR}. However, because the involvement of human body, developing control algorithms for dressing differs substantially from robotic manipulation of cloth alone. In this paper, we do not consider to solve the problems related to grasping and re-grasping, since this constitutes distinct challenges and is actively being addressed by others in the robotics community.


% Papers that demonstrate dressing:
% - Character Motion Synthesis by Topology Coordinates: They generated keyframes in topology coordinates and interpolate motion. The character does not respond to the state of the clothes. No autonomous control. They demonstrated that a character stretching the arms out of a piece clothes wrapped around it using keyframes. 
% - Harmonic Parameterization by Electrostatics
% - Clothing manipulation, Igarashi

A self-dressing virtual character has been previously demonstrated by a few methods. Ho and Kumura \shortcite{Ho:2009:CMS} introduced a technique to interact with deformable bodies using topology coordinates, in which the topological relationship of the character's body and the environment can be easily controlled. They generated keyframe animation in topology coordinates to demonstrate that a character is able to stretch her arms out of a piece of clothes wrapped around her. To demonstrate the effect of using electric flux for path planning, Wang \etal \shortcite{Wang:2013:HPE} showed a virtual human putting on a sock and a pair of shorts. In both cases, the clothes are already aligned with the body parts and the character simply needs to pull them in the direction guided by the electric flux. Neither of the previous methods attempted to control the character based on the state of the cloth. In contrast, our work designs a feedback controller such that the character can act autonomously based on the state of the cloth, without any assumption about the initial state of the cloth being in particular configurations.

% Cloth sim interacting with rigids
% - Coupling Cloth and Rigid Bodies for Dexterous Manipulation

Although cloth simulation is a relatively matured research area, dynamic coupling between cloth and rigid body systems still presents many challenges. A variety of methods are proposed to handle two-way coupling between deformable and rigid bodies \cite{Jansson:2003:CDR,Sifakis:2007:HSD,Shinar:2008:TCR,Otaduy:2009:ICH,garre2011interactive}, which can be potentially extended to rigid-cloth coupling. Otaduy et. al. \cite{Otaduy:2009:ICH} solved contacts between cloth and rigid bodies by implicitly solving a large mixed linear complementarity problem. Bai and Liu \cite{Bai:2014:CCR} proposed a simpler coupling method which treats existing cloth and rigid body simulators as black boxes without altering the internal formulation of collision handling. In this work, we directly uses the open source multibody simulator, DART \cite{Liu:2012:STM}, and the cloth simulator, ARCSim\cite{Narain:2012:AAR,Narain:2013:FCA}. ARCSim treats the rigid bodies in the scene as objects with infinite mass and only considers the contact forces from the rigid bodies to the cloth. Since the type of clothes we consider in this paper is relatively massless comparing to the human character, ignoring the impact of cloth to the character is a reasonable assumption. However, considering accurate two-way coupling might be critical for dressing tighter clothing or assistive dressing.

