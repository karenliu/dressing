Dressing is one of the most common activities in human society. Perfecting the skill of dressing can take an average child three to four years of daily practice. The challenge is primarily due to the combined difficulty in coordinating different body parts and manipulating soft and deformable objects (clothes). We present a technique to synthesize human dressing by controlling a human character to put on an article of simulated clothing. We identify a set of \emph{primitive actions} which account for the vast majority of motions observed in human dressing. These primitive actions can be assembled into a variety of motion sequences for dressing different garments with different styles. Exploiting both feedforward and feedback control mechanisms, we develop a dressing controller to handle each of the primitive actions. The controller plans a path to achieve the action goal while making constant adjustments locally based on the current state of the simulated cloth when necessary. We demonstrate that our framework is versatile and able to dress different clothing types including a jacket, a pair of shorts, a robe, and a vest. Our controller is also robust to different cloth mesh resolutions which can cause the cloth simulator to generate significantly different cloth motions. In addition, we show that the same controller can be extended to assistive dressing.

% Because the human dressing motion is difficult to animate or motion capture, the input motion does not need to be exact or complete (a few keyframes or pretend-dressing mocap). We develop a feedback controller that takes into account the state of cloth
