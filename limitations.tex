\section{Limitations}

Even though our system has produced a number of successful dressing
animations, our current approach has some limitations.  One such
limitation is due to the nature of the feedback from the cloth position.
In our system, this feedback is performed using visibility calculations.
When the control system is guiding a hand to enter a sleeve, this is done
by finding the closest point on the cloth to the sleeve entry that is
visible to the tip of the hand.  It is probable that when real people
dress themselves, much of their knowledge about the cloth state is due
to tactile feedback, instead of from visual information.

Another limitation of our dressing controller is that it uses kinematic
motion instead of calculating the dynamics of the human body.  This has
two reprecussions.  First, our system is one-way coupled, so that the
cloth does not exert forces on the human.  Certain character motions can
occasionally cause the cloth to exceed its strain limit, producing
unrealistic cloth behavior.  This problem could be eliminated using
feedback forces from the cloth.  A second consequence is that the
simulated human has no notion of balance, and thus may carry out
physically impossible motions.  This is especially important in dressing
tasks such as putting on a pair of pants while standing up.  The lack of
balance control could also have more subtle effects on the stance of the
person during upper body dressing.

In all of our examples, the initial configuration of the cloth has been
set in a way that is favorable to dressing.  If the initial garment were
tangled, our dressing actions would most likely fail.  A possible avenue
for addressing this may be found in the robotics research that has
investigated picking up and folding cloth~\cite{Cusumano:2011:BCD}.

A final limitation of our system is that it requires the user to specify
the set of actions for a given dressing task.  When given a new garment,
the system has no way of determining a sequence of actions that will
successfully place it on a body.  We can imagine a more sophisticated
system that would analyze an entirely new garment and form a plan of
actions to properly dress the character.  

% \karen{Perhaps we can add a paragraph about initial state of
%   cloth. Currently, we start the simulation with the cloth in
%   hand. Roboticists have been trying to fold laundry from random
%   initial cloth configurations.}
