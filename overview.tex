\section{Overview}

We have designed a system that allows a virtual human character to dress himself or assist another character to put on various types of garments. The
goal of this system is to provide users a simple-to-use interface to design a dressing animation with a physically simulated garment.
Given a piece of clothes and a reference human dressing motion that specifies the dressing style, our system allows a user to decompose the entire
dressing process into multiple stages and specify high level actions for each stage. For example, dressing the first sleeve of a jacket can be achieved with
the following four actions: \emph{Grip} the colar, \emph{align} the other hand with the opening of the sleeve,
\emph{drag} the cloth from the wrist to the shoulder, and \emph{idle} until the cloth is settled.
At each time step, our system simulates the dynamics of the clothes, determines the current stage of dressing and executes the corresponding low level controller
to fulfill the user-specified action. While some of the actions, such as following the reference motion, can be achieved using simple feedforward controls,
other actions require sophisticated path planning and feedback control. We apply RRT to plan motion paths that are free of inter-body collisions, and solve constrained
optimization problems to move the end effectors into the desired location without knocking away or overstreching the clothes.
Figure 2 illustrates the main components of our system. 





