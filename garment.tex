\section{Garment}

We chose four different garments from Berkeley Garment Library and made significant edits to these garment models to create a jacket, a pair of shorts, a robe, and vest that fit the size of our human character. The garments are modeled as a finite element mesh and physically simulated using ARCSim cloth simulator \cite{Narain:2012:AAR}. We use linear stretching and bending models and constitutive models derived from measurements \cite{Wang:2011}. The collisions are detected using a bounding volume hierarchy \cite{Tang:2010} and resolved with non-rigid impact zones \cite{Harmon:2008}.

\paragraph{Garment Features.} We defined a set of cloth features important for dressing control. A feature is a set of vertices on the cloth mesh. They mark the targets where the end effectors grasp or align with during the dressing process. For example, we use the vertex loop of an armhole as a feature for the hand to aim when putting the arm into a sleeve. Figure~\ref{fig:feature} shows all the features that we use for the jacket and the shorts.
